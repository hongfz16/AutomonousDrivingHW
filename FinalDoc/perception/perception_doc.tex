%!TEX program = xelatex
%!TEX TS-program = xelatex
%!TEX encoding = UTF-8 Unicode
\documentclass[12pt]{article}

\usepackage{amsmath}
\usepackage{hyperref}
\usepackage{geometry}
\usepackage{fontspec,xltxtra,xunicode}
\usepackage{graphicx}
\usepackage{setspace}
\usepackage{geometry}
\usepackage{fontspec,xltxtra,xunicode}
\usepackage{booktabs}
\usepackage{float}
\usepackage{changepage}


\defaultfontfeatures{Mapping=tex-text}
\setromanfont{Heiti SC}
\XeTeXlinebreaklocale “zh”

\title{Project1 Perception Document}
\author{洪方舟\\Student ID: 2016013259\\Email: \href{mailto:hongfz16@163.com}{hongfz16@163.com}}

\geometry{left=2cm,right=2cm,top=1cm,bottom=2cm}
\renewcommand{\baselinestretch}{1.5}

\begin{document}
  \maketitle
  \section*{一. 项目目标}
  	从LIDAR采集的点云中提取可移动物体的包围矩形,可移动物体包括车辆与行人。
  \section*{二. 运行方法}
  	将$perception/perception.cc$文件中第24行替换为运行环境项目文件夹名即可。
  \section*{三. 方法介绍}
  	下面将从点云预处理,栅格法障碍物识别,特征提取,分类四个方面介绍本项目所用方法。
  	\subsection*{1. 点云预处理}
  		点云预处理包括去除地面点和噪声点去除。去除地面点使用作业二中所使用方法,对于每一个幅角上的点,按照距离原点远近排序后,遍历所有点,依据连续两点连线与地面夹角的大小来判断是否为障碍点。噪声点去除方法较为简易,对于每一个点周围固定范围内点数小于一定的阈值则判定为噪声点。
  	\subsection*{2. 栅格法障碍物识别}
  		将以本车为原点,30米范围内所有点归入$100 \times 100$的栅格中,对于每一格统计最大高度与最小高度之差,如果大于一定阈值则标记为障碍点。标定完成后,对栅格进行连通块搜索,同一连通块标记为一个障碍物。取包围障碍物内所有点的最小包围矩形。
  	\subsection*{3. 特征提取}
  		对于上一步骤提取出的障碍物内所有点提取特征\cite{1},主要包括如下特征
  		 \begin{table}[h!]
  			\begin{center}
  				\caption{三维点云特征描述}
  				\begin{tabular}{c|c}
  					\toprule
  					\textbf{特征描述} & \textbf{维数}\\
  					\midrule
  					点个数 & 1\\
  					惯性张量矩阵 & 6\\
  					协方差矩阵 & 6\\
  					协方差矩阵特征值 & 3\\
  					Lalonde 特征 & 12\\
  					Anguelov 特征 & 12\\
  					\bottomrule
  				\end{tabular}
  			\end{center}
  		\end{table}
  		其中Lalonde特征为参考Lalonde等人\cite{2}提出的一种基于局部特征的直方图统计量。对于每个点周围的20个邻近点,计算其协方差矩阵特征值,从大到小排序后为$\lambda_0, \lambda_1, \lambda_2$,取$L_1=\lambda_0,L_2=\lambda_0-\lambda_1,L_3=\lambda_1-\lambda_2$。对于所有的$L_1,L_2,L_3$求分组为4的直方图统计,因此一共是12维。\\
  		Anguelov特征为参考Anguelov等人\cite{3}提出的一种基于局部特征的直方图统计量。对于每个点取周围直径为0.1米,母线长2米的圆柱体内所有点,将所有点从上至下等距离划分为三个部分,统计每个部分的数量$A_1,A_2,A_3$,并做归一化处理。对于所有$A_1,A_2,A_3$求分组为4的直方图统计,因此该特征共12维。
  	\subsection*{4. 分类}
  		本项目中使用SVM作为分类器。使用了SVM-Light项目\cite{4}的源代码。将已打标的数据提取特征后使用SVM训练,训练好的模型已经放入项目文件夹中留作在线分类使用。在线分类时,提取特征后使用SVM给出预测分类。
  \section*{四. 结果}
  	使用本课程提供工具测试结果如下:准确率43.05\%,召回率27.71\%。
  \begin{thebibliography}{3}
 	\bibitem{1}Himmelsbach, M., Mueller, A., Lüttel, T., \& Wünsche, H. J. (2008, October). LIDAR-based 3D object perception. In Proceedings of 1st international workshop on cognition for technical systems (Vol. 1).
  	\bibitem{2}Lalonde, J. F., Vandapel, N., Huber, D. F., \& Hebert, M. (2006). Natural terrain classification using three‐dimensional ladar data for ground robot mobility. Journal of field robotics, 23(10), 839-861.
  	\bibitem{3}Anguelov, D., Taskarf, B., Chatalbashev, V., Koller, D., Gupta, D., Heitz, G., \& Ng, A. (2005, June). Discriminative learning of markov random fields for segmentation of 3d scan data. In Computer Vision and Pattern Recognition, 2005. CVPR 2005. IEEE Computer Society Conference on (Vol. 2, pp. 169-176). IEEE.
  	\bibitem{4}Joachims, T. (1999). Svmlight: Support vector machine. SVM-Light Support Vector Machine http://svmlight. joachims. org/, University of Dortmund, 19(4).
  \end{thebibliography}
 \end{document}