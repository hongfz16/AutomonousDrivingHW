%!TEX program = xelatex
%!TEX TS-program = xelatex
%!TEX encoding = UTF-8 Unicode
\documentclass[12pt]{article}

\usepackage{amsmath}
\usepackage{hyperref}
\usepackage{geometry}
\usepackage{fontspec,xltxtra,xunicode}
\usepackage{graphicx}
\usepackage{setspace}
\usepackage{geometry}
\usepackage{fontspec,xltxtra,xunicode}
\usepackage{booktabs}
\usepackage{float}
\usepackage{changepage}


\defaultfontfeatures{Mapping=tex-text}
\setromanfont{Heiti SC}
\XeTeXlinebreaklocale “zh”

\title{Project2 PNC Document}
\author{洪方舟\\Student ID: 2016013259\\Email: \href{mailto:hongfz16@163.com}{hongfz16@163.com}}

\geometry{left=2cm,right=2cm,top=1cm,bottom=2cm}
\renewcommand{\baselinestretch}{1.5}

\begin{document}
  \maketitle
  \section*{一. 项目目标}
  	实现自动驾驶汽车规划以及控制算法,在遵守交通规则、不撞到行人以及其他车、尽量保证乘客舒适的前提下,在模拟环境中完成10趟接送客人的任务。
  \section*{二. 运行方法}
  	将$hongfz16/hongfz16_agent.h$文件中第166行替换为运行环境项目文件夹名即可。
  \section*{三. 方法介绍}
	下面将从路径规划,速度规划,方向控制,速度控制四个方面阐述该项目所使用的方法。
	\subsection*{1. 路径规划}
		路径规划发生在新的行程请求到达之后。路径规划算法利用预处理的地图中前驱后继信息,使用广度优先搜索找到最短路径。将途经所有道路中心线上点存入路径信息。
	\subsection*{2. 速度规划}
		速度规划发生于每一次迭代中。基本思想为根据路况计算出某个预计点所需要到达的速度,然后根据该速度,预计点与自身距离计算出期望加速度。路况信息主要分为三类:可移动障碍物(行人与车辆),红绿灯,到达终点。对于可移动障碍,计算其行进路线与自身规划路径是否存在交点,如果有,则以该交点为预计点,根据可移动障碍到该交点的距离计算该点的期望速度。对于红绿灯与到达终点的情况,则提前一定的距离开始减速,以红绿灯路口或终点为预计点,期望速度为0。每一次迭代中给出当前期望加速度,供速度控制模块使用。
	\subsection*{3. 方向控制}
		方向控制使用改进的PID算法。由于单一误差量很难准确描述车辆状态,因此本项目采用两种误差量:当前车辆位置到期望路径的距离(Cross Track Error),当前速度方向与期望行进方向的角度差。分别对上述两个误差量进行PID控制,将控制结果相加返回作为最终方向盘转向控制量。
	\subsection*{4. 速度控制}
		速度控制相对方向控制较为简单,直接对期望加速度与当前加速度之差进行PID控制。
  \section*{四. 流程介绍}
  	1. 每次迭代中首先检查是否有新的路径请求,如果有,则进行一系列初始化操作,包括路径规划。\\
  	2. 如果没有新的路径请求,则开始如下控制流程。首先进行方向控制计算两个误差量,经过PID控制得到方向盘命令。\\
  	3. 接着进行速度控制,通过速度规划得到期望加速度,通过速度控制得到加速或减速命令。\\
  	4. 返回命令,一次迭代结束。\\
  \section*{五. 结果}
  	在大多数情况下能够顺利完成任务。一次典型的运行结果如下。
  	\begin{align*}
  		num\_finish\_trips&: \: 10\\
  		big\_acc\_sqr\_sum\_total&: \: 138859.2208884063\\
  		big\_acc\_sqr\_sum\_per\_trip&: \: 13885.922088840631\\
  		curvature\_sqr\_sum\_total&: \: 55593133.817445092\\
  		curvature\_sqr\_sum\_per\_trip&: \: 5559313.38174451\\
  		time\_cost\_total&: \: 979.88999999925386\\
  		time\_cost\_per\_trip&: \: 97.988999999925383\\
  	\end{align*}
 \end{document}